\documentclass[12pt, titlepage, twoside]{amsart}

\usepackage[a4paper, margin=1in]{geometry}
\usepackage{amsmath}
\usepackage[foot]{amsaddr}
\usepackage{amssymb}
\usepackage{amsthm}
\usepackage{enumitem}
\usepackage[dvipsnames]{xcolor}
\usepackage{parskip}
\usepackage{graphicx}
\usepackage{tikz}
\usepackage{listings}
\usepackage{lipsum}
\usepackage[cache=false]{minted}
\usepackage{hyperref}

\newcommand{\R}{\ensuremath{\mathbb R}}
\newcommand{\Z}{\ensuremath{\mathbb Z}}
\newcommand{\N}{\ensuremath{\mathbb N}}
\newcommand{\C}{\ensuremath{\mathbb C}}

\hypersetup{
  colorlinks=true,
  linkcolor=Orchid,
  urlcolor=ProcessBlue
}

\usemintedstyle{stata-dark}

\raggedright

\begin{document}

\title[ScottyRank.jl]{ScottyRank.jl: An Implementation of PageRank \& HITS}

\author{Siyuan Chen}
\author{Michael Zhou}
\email{siyuanc2@andrew.cmu.edu}
\email{mhzhou@andrew.cmu.edu}
\date{November 2021}

\maketitle

\section{Mathematical Background}

\subsection{Definitions}

Positive matrices are defined as matrices with positive entries.

Markov matrices are defined as square matrices with nonnegatives entries and column sum $1$ across all of its columns.
Note that for a $n\times n$ matrix $M$, the latter condition is equivalent to $M^T\vec{1} = \vec{1}$,
where $\vec{1}\in\R^n$ has all ones as components.

Positive Markov matrices are defined as, well, positive Markov matrices.

\subsection{Facts}

(Perron-Frobenius theorem)
Let $A$ be a positive square matrix.
Let $\lambda_1$ be $A$'s maximum eigenvalue in terms of absolute values.
Then $\lambda_1$ is positive and has algebraic (and subsequently geometric) multiplicity $1$.

Let $M$ be a Markov matrix.
Let $\lambda_1$ be $M$'s maximum eigenvalue in terms of absolute values.
Then $\lambda_1 = 1$.

Let $M'$ be a positive Markov matrix.
Let $\lambda_1$ be $M'$'s maximum eigenvalue in terms of absolute values.
Then $\lambda_1 = 1$ and has algebraic (and subsequently geometric) multiplicity $1$.

\subsection{Usage}

Let $M$ be a $n\times n$ Markov matrix.
Then $M$ specifies a dicrete memoryless transition process between $n$ states, namely the process where
\[
  \left(\forall (t, i, j)\in\N\times[n]\times[n]\right)
  \left[\Pr(\text{state }i\text{ at time }t + 1\mid \text{state }j\text{ at time }t) = M_{ij}\right].
\]

Let $\vec{v}\in\R^n$ such that $\vec{v}$ has nonnegative components and $\vec{v}^T\vec{1} = 1$ (a stochastic vector).
Then $\vec{v}$ specifies an (initial) discrete probability distribution over the $n$ states, namely the distribution where
\[
  (\forall i\in[n])
  [\Pr(\text{state }i\text{ at time }0) = \vec{v}_i].
\]

Then
\[
  (\forall (k, i)\in\N\times[n])
  \left[\Pr(\text{state }i\text{ at time }k) = \left(M^k\vec{v}\right)_i\right].
\]

\end{document}
