\documentclass[12pt, titlepage, twoside]{amsart}

\usepackage[a4paper, margin=1in]{geometry}
\usepackage{amsmath}
\usepackage[foot]{amsaddr}
\usepackage{amssymb}
\usepackage{amsthm}
\usepackage{enumitem}
\usepackage[dvipsnames]{xcolor}
\usepackage{parskip}
\usepackage{graphicx}
\usepackage{tikz}
\usepackage{listings}
\usepackage{lipsum}
\usepackage{minted}
\usepackage{hyperref}

\newcommand{\R}{\ensuremath{\mathbb R}}
\newcommand{\Z}{\ensuremath{\mathbb Z}}
\newcommand{\N}{\ensuremath{\mathbb N}}
\newcommand{\C}{\ensuremath{\mathbb C}}

\hypersetup{
  colorlinks=true,
  linkcolor=Orchid,
  urlcolor=ProcessBlue
}

\setminted{linenos, breaklines}

\raggedright

\begin{document}

\title[ScottyRank.jl]{ScottyRank.jl: An Implementation of PageRank \& HITS}

\author{Siyuan Chen}
\author{Michael Zhou}
\email{siyuanc2@andrew.cmu.edu}
\email{mhzhou@andrew.cmu.edu}
\date{November 2021}

\maketitle

\section{Mathematical Background}

\subsection{Linear Algebra}

\subsubsection{Definitions}

Positive matrices are defined as matrices with positive entries.

Markov matrices are defined as square matrices with nonnegatives entries and column sum $1$ across all of its columns.
Note that for a $n\times n$ matrix $M$, the latter condition is equivalent to $M^T\vec{1} = \vec{1}$,
where $\vec{1}\in\R^n$ has all ones as components.

Positive Markov matrices are defined as, well, positive Markov matrices.

\subsubsection{Facts}

(Perron-Frobenius theorem)
Let $A$ be a positive square matrix.
Let $\lambda_1$ be $A$'s maximum eigenvalue in terms of absolute values.
Then $\lambda_1$ is positive and has algebraic (and subsequently geometric) multiplicity $1$.

Let $M$ be a Markov matrix.
Let $\lambda_1$ be $M$'s maximum eigenvalue in terms of absolute values.
Then $\lambda_1 = 1$.

Let $M'$ be a positive Markov matrix.
Let $\lambda_1$ be $M'$'s maximum eigenvalue in terms of absolute values.
Then $\lambda_1 = 1$ and has algebraic (and subsequently geometric) multiplicity $1$.

\subsubsection{Usage}

Let $M$ be a $n\times n$ Markov matrix.
Then $M$ specifies a dicrete memoryless transition process between $n$ states, namely the process where
\[
  \left(\forall (t, i, j)\in\N\times[n]\times[n]\right)
  \left[\Pr(\text{state }i\text{ at time }t + 1\mid \text{state }j\text{ at time }t) = M_{ij}\right].
\]

Let $\vec{v}\in\R^n$ such that $\vec{v}$ has nonnegative components and $\vec{v}^T\vec{1} = 1$ (a stochastic vector).
Then $\vec{v}$ specifies an (initial) discrete probability distribution over the $n$ states, namely the distribution where
\[
  (\forall i\in[n])
  [\Pr(\text{state }i\text{ at time }0) = \vec{v}_i].
\]

Then the probability distribution over the $n$ states after $t$ steps of the transition process specified by $M$ is precisely $M^t\vec{v}$,
or equivalently
\[
  (\forall (t, i)\in\N\times[n])
  \left[\Pr(\text{state }i\text{ at time }t) = \left(M^t\vec{v}\right)_i\right].
\]

\subsection{Graph Theory}

\subsubsection{Definitions}

A simple directed graph is defined as an unweighted directed graph without self-referential edges or multiple edges
between the same origin destination pair.

For a simple directed graph with $n$ vertices, the adjacency matrix $\mathcal{A}$ is defined to be
the $n\times n$ matrix where
\[
  (\forall (i, j)\in[n]\times[n])\left(A_{ij} = 
    \begin{cases}
      1 & \text{there is an edge to $i$ from $j$} \\
      0 & \text{otherwise}
    \end{cases}
  \right).
\]

\subsubsection{Facts}

For a simple directed graph with $n$ vertices and its adjacency matrix $\mathcal{A}$,
\begin{align*}
  (\forall j\in[n])&
  \left[\text{number of outgoing neighbors from vertex $j$} = \mathrm{out}(j) = (\mathcal{A}_{*j})^T\vec{1}\right] \\
  (\forall i\in[n])&
  \left[\text{number of incoming neighbors to vertex $i$} = \mathrm{in}(i) = (\mathcal{A}_{i*})^T\vec{1}\right].
\end{align*}

\section{Algorithms}

\subsection{The network model}

Both algorithms, PageRank and HITS, model the network of interest as a simple directed graph with websites as vertices
and links as edges.
This implies that there will be no self-referential links, no duplicate links between the same origin
and destination pair, and no priority difference between links.

\subsection{PageRank}

\subsubsection{The random walk}

PageRank models the behavior of a typical web surfer as a damped random walk.

\begin{enumerate}
  \item The surfer starts out by visiting a random site out of all sites with equal probability.
  \item At every step, the surfer has a probability $\lambda$ of continuing surfing and a complementary
    $1 - \lambda$ probability of losing interest, for a predetermined $\lambda$.
  
    \begin{enumerate}
      \item If the surfer continues \ldots

        \begin{enumerate}
          \item \ldots and there are links exiting the current site, the surfer
            clicks on a random link (and visits the site it points to) out of those links with equal probability.
          \item \ldots and there aren't any links exiting the current site, the surfer simply
            visits a random site out of all sites with equal probability.
        \end{enumerate}

      \item If the surfer loses interest, they simply visits a random site out of all sites with equal probability.
    \end{enumerate}

\end{enumerate}

To best model a typical surfer's probability of continuing surfing, $\lambda$, also known as the damping factor,
is empirically determined to be around $0.85$.

\subsubsection{Matrix representation}

Let $n$ be the number of websites in the network of interest.
Let $\mathcal{A}$ be the adjacency matrix for the network of interest.
Let $\langle\vec{v}_{t}\rangle_{t\in\N}$ be the probability distributions describing the
website the surfer is visiting at time $t$.
Let $M$ be the transition matrix for the random walk process.

Then $\vec{v}_0 = \vec{1} / n$,
$M$ is the $n\times n$ matrix where
\[
  (\forall (i, j)\in[n]\times[n])
  \left[
  M_{ij} =
  \begin{cases}
    \frac{\lambda}{\mathrm{out}(j)} + \frac{1 - \lambda}{n}
    & \mathcal{A}_{ij} = 1 \\
    \frac{1 - \lambda}{n}
    & \mathcal{A}_{ij} = 0\wedge\mathrm{out}(j) > 0 \\
    \frac{\lambda}{n} + \frac{1 - \lambda}{n}
    & \mathrm{out}(j) = 0
  \end{cases}
  \right],
\]
and
\[
  (\forall t\in\N)\left(\vec{v}_t = M^t\vec{v}_0\right).
\]

Note that in this case $M$ is a positive Markov matrix, assuming reasonable $\lambda$.

\subsubsection{Definition}

The PageRank score for a given website in the network of interest is defined as the probabilty of a typical surfer
visiting that website after an indefinitely long damped random walk.
In matrix form,
\[
  (\forall i\in[n])
  \left[\mathrm{PageRank}(i) = \lim_{t\to\infty}(\vec{v}_t)_i = \lim_{t\to\infty}\left(M^t\vec{v}_0\right)_i\right].
\]

Note that the limits exist: convergence is guaranteed as $M$ has a unique maximal eigenvalue of $1$ and thus an steady
attracting state.

\subsection{HITS}

\subsubsection{Authorities and hubs}

filler

\subsubsection{Matrix representation}

filler

\begin{minted}{julia}
# export read_graph

"""
    read_graph(filepath::String="data/medium-el.txt"); filetype::String="el", zero_index::Bool=false) -> Graph

Reads a graph from an edge list/adjacency list file

# Arguments
- `filepath::String="data/medium-el.txt"`: the path to the source file (default: Wikipedia PageRank graph)

# Keywords
- `filetype::String="el"`: "el" for edge list, "al" for adjacency list
- `zero_index::Bool=false`: whether the input file is zero-based

# Returns
- `Graph`: the graph from the source file
"""
function read_graph(filepath::String="data/medium-el.txt";
    filetype::String="el", zero_index::Bool=false)
  if filetype == "el"
    read_edge_list(filepath, zero_index)
  elseif filetype == "al"
    read_adjacency_list(filepath, zero_index)
  else
    error("invalid filetype")
  end
end
\end{minted}

\end{document}
